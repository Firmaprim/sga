% !TEX root = SGA-1.tex
\usepackage{hyperref}
\usepackage{xcolor}
\hypersetup{colorlinks,linkcolor={blue!50!black},citecolor={blue!50!black},urlcolor={blue!80!black}}
\usepackage{amsmath,amssymb}
\usepackage{mathrsfs}
\usepackage{enumerate}
\usepackage{fouriernc} % not necessary, but looks nice :) %


%% Headers and footers and titles
\usepackage{fancyhdr}
\usepackage{xstring}
\renewcommand{\chaptername}{}
\pagestyle{fancy}
\fancypagestyle{plain}{}
\fancyhf{}
\lhead{\footnotesize\nouppercase\leftmark}
\cfoot{\small\thepage}


%% Theorem environments
\usepackage{amsthm}
\newenvironment{itenv}[1]
  {\phantomsection\par\medskip\noindent\textbf{#1.}\itshape}
  {\par\medskip}
\newenvironment{rmenv}[1]
  {\phantomsection\par\medskip\noindent\textbf{#1.}\rmfamily}
  {\par\medskip}


% COMMANDS

\renewcommand{\geq}{\geqslant}
\renewcommand{\leq}{\leqslant}

%% Shortcuts
\newcommand{\kres}{\kappa}
\newcommand{\id}{\mathrm{id}}

\DeclareMathOperator{\Spec}{Spec}

%% "Functions"
\renewcommand{\cal}[1]{{\mathcal{#1}}}
\renewcommand{\frak}[1]{{\mathfrak{#1}}}
\newcommand{\scr}[1]{{\mathscr{#1}}}
\newcommand{\cat}[1]{{\mathsf{#1}}}

%% Non-mathematical
\newcommand{\todo}[1]{\textbf{! #1 !}}
\usepackage{marginnote}
\newcommand{\oldpage}[1]{\marginparmargin{left}\marginnote[\footnotesize\textit{p.~#1}~$\Big\vert$]{\footnotesize$\Big\vert$~\textit{p.~#1}}}
