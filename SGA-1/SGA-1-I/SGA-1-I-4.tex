%!TEX root = ../SGA-1.tex
% \section{\'{E}tale morphisms. \'{E}tale coverings}
% \label{I.4}

% We are going to suppose all that will be necessary concerning flat morphisms; these facts will be proved later, if there is time\footnote{Cf. Exp.~IV.}.

% \begin{defn}
%     \begin{enumerate}[\normalfont a)]
%         \item Let $f\colon X\to Y$ be a morphism of finite type.
%             We say that $f$ is \emph{étale} at $x$ if $f$ is both flat and unramified at $x$.
%             We say that $f$ is étale if it is étale at all points.
%             We say that $X$ is étale at $x$ over $Y$, or that it is a $Y$-prescheme étale at $x$ etc.
%         \item Let $f\colon A\to B$ be a local homomorphism.
%             We say that $f$ is étale, or that $B$ is étale over $A$, if $B$ is flat and unramified over $A$\footnote{Cf. regrets in III~1.2.}.
%     \end{enumerate}
% \end{defn}

% \begin{prop}
%     For $B/A$ to be étale, it is necessary and sufficient that $\hat{B}/\hat{A}$ be étale.
% \end{prop}

% \oldpage{5}In effect, this is true individually for both ``unramified'' and ``flat''.

% \begin{cor}
%     Let $f\colon X\to Y$ be of finite type, and $x\in X$.
%     The property of $f$ being étale at $x$ depends only on the local homomorphism $\O_{f(x)}\to\O_x$, and in fact only on the corresponding homomorphism for the completions.
% \end{cor}

% \begin{cor}
%     Suppose that the residue extension $k(A)\to k(B)$ is trivial, or that $k(A)$ is algebraically closed.
%     Then $B/A$ is étale if and only if $\hat{A}\to\hat{B}$ is an isomorphism.
% \end{cor}

% We combine the platitude and 3.7.

% \begin{prop}
%     Let $f\colon X\to Y$ be a morphism of finite type.
%     Then the set of points where $f$ is étale is open.
% \end{prop}
% Again, this is in fact true individually for both ``unramified'' and ``flat''.

% This proposition shows that we can forget about the punctual comments in the study of morphisms of finite type that are somewhere étale.

% \begin{prop}
%     \begin{enumerate}[\normalfont(i)]
%         \item An open immersion is étale.
%         \item The composition of two étale morphisms is étale.
%         \item The base extension of an étale morphism is étale.
%     \end{enumerate}
% \end{prop}

% In effect, (i) is trivial, and for (ii) and (iii) it suffices to note that it is true for ``unramified'' and ``flat''.
% As a matter of fact, there are also corresponding comments to make about local homomorphisms (without the finiteness condition), which in any case should figure in the \marginpar{\small Grothendieck's \emph{multiplodoque d'algèbre homologique} was the final version of his \emph{Tohoku paper} — see (2.1) in `Life and work of Alexander Grothendieck' by Ching-Li Chan and Frans Oort for more information. [trans.]}multiplodoque (starting with the case of unramified).

% \begin{cor}
%     A cartesian product of two étale morphisms is étale.
% \end{cor}

% \begin{cor}
%     Let $X$ and $X'$ be of finite type over $Y$, and $g\colon X\to X'$ a $Y$-morphism.
%     If $X'$ is unramified over $Y$ and $X$ is étale over $Y$, then $g$ is étale.
% \end{cor}

% In effect, $g$ is the composition of the graph morphism $\Gamma_g\colon X\to X\times_Y X'$ (which is an open immersion by 3.4) and the projection morphism, which is étale since it is just a ``change of base'' by $X'\to Y$ of the étale morphism $X\to Y$.

% \begin{defn}
%     We say that a cover of $Y$ is étale (resp. unramified) if it is a $Y$-scheme $X$ that is finite over $Y$ and étale (resp. unramified) over $Y$.
% \end{defn}
% The first condition means that $X$ is defined by a coherent sheaf of algebras $\mathscr{B}$ over $Y$.
% The second means that $\mathscr{B}$ is locally free over $Y$ (resp. \unsure{absolutely nothing}) \emph{and} further that, for all $y\in Y$, the fibre $\mathscr{B}(y)=\mathscr{B}_y\otimes_{\O_y}k(y)$ is a separable algebra (i.e. a finite composition of finite separable extensions) over $k(y)$.

% \begin{prop}
%     Let $X$ be a flat cover of $Y$ of degree $n$ (the definition of this term deserved to figure in 4.9) defined by a locally free coherent sheaf $\mathscr{B}$ of algebras.
%     We define, as usual, the trace homomorphism $\mathscr{B}\to\mathscr{A}$ (that is a homomorphism of $\mathscr{A}$-modules, where $\mathscr{A}=\O_Y$).
%     For $X$ to be étale it is necessary and sufficient that the corresponding bilinear form $\tr_{\mathscr{B}/\mathscr{A}}xy$ defines an isomorphism of $\mathscr{B}$ over $\mathscr{B}$, or, equivalently, that the \emph{discriminant section}
%     \begin{equation*}
%         d_{X/Y}=d_{\mathscr{B}/\mathscr{A}}\in\Gamma(Y,\wedge^n\check{\mathscr{B}}\otimes_\mathscr{A}\wedge^n\check{\mathscr{B}})
%     \end{equation*}
%     is invertible, or that the discriminant ideal defined by this section is the unit ring.
% \end{prop}

% In effect, we can reduce to the case where $Y=\Spec(k)$, and then it is a well-known criterion of separability, and trivial by passing to the algebraic closure of $k$.

% \begin{rem}
%     We will have a less trivial statement to make later on, \oldpage{6}when we do not suppose a priori that $X$ is flat over $Y$, but instead require some normality hypothesis.
% \end{rem}