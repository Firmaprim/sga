%!TEX root = ../SGA-1.tex
\section{Basics of differential calculus}
\label{I.1}

Let $X$ be a prescheme on $Y$, and $\Delta_{X/Y}$ the diagonal morphism $X\to X\times_Y X$.
This is an immersion, and thus a closed immersion of $X$ into an open subset $V$ of $X\times_Y X$.
Let $\scr{I}_X$ be the ideal of the closed sub-prescheme corresponding to the diagonal in $V$ (N.B. if one really wishes to do things intrinsically, without assuming that $X$ is separated over $Y$ --- a misleading hypothesis --- then one should consider the set-theoretic inverse image of $\cal{O}_{X\times X}$ in $X$ and denote by $\scr{I}_X$ the augmentation ideal in the above \ldots).
The sheaf $\scr{I}_X/\scr{I}_X^2$ can be thought of as a quasi-coherent sheaf on $X$, which we denote by $\Omega_{X/Y}^1$.
This sheaf is of finite type if $X\to Y$ is of finite type, and it behaves well with respect to a base change $Y'\to Y$.
We also introduce the sheaves $\cal{O}_{X\times_Y X}/\scr{I}_X^{n+1}=\scr{P}^n_{X/Y}$, which are sheaves of \emph{rings} on $X$, giving us preschemes denoted by $\Delta_{X/Y}^n$ and called the \emph{$n$-th infinitesimal neighbourhood of $X/Y$}.
The polysyllogism is entirely trivial, even if rather long\footnote{cf. EGA~IV~16.3.}; it seems wise to not discuss it until we use it for something helpful, with smooth morphisms.
