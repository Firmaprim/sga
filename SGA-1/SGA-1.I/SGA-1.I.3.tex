%!TEX root = ../SGA-1.tex
\section{Unramified morphisms}
\label{I.3}

\begin{itenv}{Proposition 3.1}
  Let $f\colon X\to Y$ be a morphism of finite type, $x\in X$, and $y=f(x)$.
  Then the following conditions are equivalent:
  \begin{enumerate}[(i)]
    \item $\cal{O}_x/\frak{m}_y\cal{O}_x$ is a finite separable extension of $\kres(y)$.
    \item $\Omega_{X/Y}^1$ is zero at $x$.
    \item The diagonal morphism $\Delta_{X/Y}$ is an open immersion on a neighbourhood of $x$.
  \end{enumerate}
\end{itenv}

\begin{proof}
  For the implication (i) $\implies$ (ii), we can use Nakayama to reduce to the case where $Y=\Spec(k)$ and $X=\Spec(k')$, where it is well known, and also trivial by the definition of separable;
  (ii) $\implies$ (iii) comes from a nice and easy characterisation of open immersions, using Krull;
  (iii) $\implies$ (i) follows as well from reducing to the case where $Y=\Spec(k)$ and the diagonal morphism is everywhere an open immersion.
  We must then prove that $X$ is finite with separable ring over $k$, and this leads us to consider the case where $k$ is algebraically closed.
  But then every closed point of $X$ is isolated (since it is identical to the inverse image of the diagonal by the morphism $X\to X\times_k X$ defined by $x$), whence $X$ is finite.
  We can thus suppose that $X$ consists of a single point, with ring $A$, and so $A\otimes_k A\to A$ is an isomorphism, hence $A=k$.
\end{proof}

\begin{rmenv}{Definition 3.2}
  \begin{enumerate}[(a)]
    \item We then say that $f$ is \emph{unramified} at $x$, or that $X$ is unramified at $x$ on $Y$.
    \item Let $A\to B$ be a local homomorphism.
        We say that it is \emph{unramified}, or that $B$ is a local \emph{unramified} algebra on $A$, if $B/\frak{r}(A)B$ is a finite separable extension of $A/\frak{r}(A)$, i.e. if $\frak{r}(A)B=\frak{r}(B)$ and $\kres(B)$ is a separable extension of $\kres(A)$\footnote{Cf. regrets in III~1.2.}.
  \end{enumerate}
\end{rmenv}

\oldpage{3}
\begin{rmenv}{Remarks}
    The fact that $B$ is unramified over $A$ can be seen at the level of the completions of $A$ and $B$.
    Unramified implies quasi-finite.
\end{rmenv}

\begin{itenv}{Corollary 3.3}
    The set of points where $f$ is unramified is open.
\end{itenv}

\begin{itenv}{Corollary 3.4}
    Let $X'$ and $X$ be preschemes of finite type over $Y$, and $g\colon X'\to X$ a $Y$-morphism.
    If $X$ is unramified over $Y$, then the graph morphism $\Gamma_g\colon X'\to X\times_Y X$ is an open immersion.
\end{itenv}

Indeed, this is the inverse image of the diagonal morphism $X\to X\times_Y X$ by
\[
  g\times_Y \id_{X'}\colon X'\times_Y X\to X\times_Y X.
\]
One can also introduce the annihilator ideal $\frak{d}_{X/Y}$ of $\Omega_{X/Y}^1$, called the \emph{different ideal} of $X/Y$;
it defines a closed sub-prescheme of $X$ which, set-theoretically, is the set of point where $X/Y$ is ramified, i.e. not unramified.

\begin{itenv}{Proposition 3.5}
  \begin{enumerate}[(i)]
    \item An immersion is ramified.
    \item The composition of two ramified morphisms is also ramified.
    \item Base extension of a ramified morphisms is also ramified.
  \end{enumerate}
\end{itenv}

We are rather indifferent about (ii) and (iii) (the second seems more interesting to me).
We can, of course, also be more precise, by giving some one-off statements;
this is more general only in appearance (except for in the case of definition~(b)), and is boring.
We obtain, as per usual, the corollaries:

\begin{itenv}{Corollaries 3.6}
  \begin{enumerate}[(i)]
    \setcounter{enumi}{3}
    \item The cartesian product of two unramified morphisms is unramified.
    \item If $gf$ is unramified then so too is $f$.
    \item If $f$ is unramified then so too is $f_\text{red}$.
  \end{enumerate}
\end{itenv}

\begin{itenv}{Proposition 3.7}
  Let $A\to B$ be a local homomorphism, and suppose that the residue extension $\kres(B)/\kres(A)$ is trivial, with $\kres(A)$ algebraically closed.
  In order for $B/A$ to be unramified, it is necessary and sufficient that $\widehat{B}$ be (as an $\widehat{A}$-algebra) a quotient of $\widehat{A}$.
\end{itenv}

\begin{rmenv}{Remarks}
  \begin{itemize}
    \item In the case where we don't suppose that the residue extension is trivial, we can reduce to the case where it is by taking a suitable finite flat extension of $A$ which destroys the aforementioned extension.
    \item Consider the example where $A$ is the local ring of an ordinary double point of a curve, and $B$ a point of its normalisation:
      then $A\subset B$, $B$ is unramified over $A$ with trivial residue extension,
\oldpage{4}
      and $\widehat{A}\to\widehat{B}$ is surjective but \emph{not injective}.
    We are thus going to strengthen the notion of unramified-ness.
  \end{itemize}
\end{rmenv}
