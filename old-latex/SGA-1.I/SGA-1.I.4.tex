%!TEX root = ../SGA-1.tex
\section{Étale morphisms. Étale coverings}
\label{I.4}

We are going to suppose that everything concerning flat morphisms that we need to be true is indeed true;
these facts will be proved later, if there is time\footnote{cf. Exp.~IV.}.

\begin{rmenv}{Definition 4.1}
  \begin{enumerate}[(a)]
    \item Let $f\colon X\to Y$ be a morphism of finite type.
      We say that $f$ is \emph{étale} at $x$ if $f$ is both flat and unramified at $x$.
      We say that $f$ is étale if it is étale at all points.
      We say that $X$ is étale at $x$ over $Y$, or that it is a $Y$-prescheme which is étale at $x$ etc.
    \item Let $f\colon A\to B$ be a local homomorphism.
      We say that $f$ is étale, or that $B$ is étale over $A$, if $B$ is flat and unramified over $A$\footnote{cf. regrets in III~1.2.}.
  \end{enumerate}
\end{rmenv}

\begin{itenv}{Proposition 4.2}
  For $B/A$ to be étale, it is necessary and sufficient that $\hat{B}/\hat{A}$ be étale.
\end{itenv}

\oldpage{5}
\begin{proof}
  Indeed, this is true individually for both ``unramified'' and ``flat''.
\end{proof}

\begin{itenv}{Corollary 4.3}
  Let $f\colon X\to Y$ be of finite type, and $x\in X$.
  The property of $f$ being étale at $x$ depends only on the local homomorphism $\cal{O}_{f(x)}\to\cal{O}_x$, and in fact only on the corresponding homomorphism for the completions.
\end{itenv}

\begin{itenv}{Corollary 4.4}
  Suppose that the residue extension $k(A)\to k(B)$ is trivial, or that $k(A)$ is algebraically closed.
  Then $B/A$ is étale if and only if $\hat{A}\to\hat{B}$ is an isomorphism.
\end{itenv}

We can combine flatness with \hyperref[I.3.7]{3.7}.

\begin{itenv}{Proposition 4.5}
\label{I.4.5}
  Let $f\colon X\to Y$ be a morphism of finite type.
  Then the set of points where $f$ is étale is open.
\end{itenv}

\begin{proof}
  Again, this is in fact true individually for both ``unramified'' and ``flat''.
\end{proof}

This proposition shows that we can forget about the ``one-off'' comments in the study of morphisms of finite type that are somewhere étale.

\begin{itenv}{Proposition 4.6}
\label{I.4.6}
  \begin{enumerate}[(i)]
    \item An open immersion is étale.
    \item The composition of two étale morphisms is étale.
    \item The base extension of an étale morphism is étale.
  \end{enumerate}
\end{itenv}

\begin{proof}
  Indeed, (i) is trivial, and for (ii) and (iii) it suffices to note that it is true for ``unramified'' and ``flat''.
\end{proof}

As a matter of fact, there are also corresponding comments to make about local homomorphisms (without the finiteness condition), which in any case should appear in the multiplodoque\translator{Grothendieck's \emph{multiplodoque d'algèbre homologique} was the final version of his \emph{Tohoku paper} --- see (2.1) in `Life and work of Alexander Grothendieck' by Ching-Li Chan and Frans Oort for more information.}
(starting with the case of unramified).

\begin{itenv}{Corollary 4.7}
\label{I.4.7}
  The cartesian product of two étale morphisms is étale.
\end{itenv}

\begin{itenv}{Corollary 4.8}
\label{I.4.8}
  Let $X$ and $X'$ be of finite type over $Y$, and $g\colon X\to X'$ a $Y$-morphism.
  If $X'$ is unramified over $Y$ and $X$ is étale over $Y$, then $g$ is étale.
\end{itenv}

\begin{proof}
  Indeed, $g$ is the composition of the graph morphism $\Gamma_g\colon X\to X\times_Y X'$ (which is an open immersion by \hyperref[I.3.4]{3.4}) and the projection morphism, which is étale since it is just a ``change of base'' by $X'\to Y$ of the étale morphism $X\to Y$.
\end{proof}

\begin{rmenv}{Definition 4.9}
\label{I.4.9}
  We say that a covering of $Y$ is étale (resp. unramified) if it is a $Y$-scheme $X$ that is finite over $Y$ and étale (resp. unramified) over $Y$.
\end{rmenv}

The first condition means that $X$ is defined by a coherent sheaf of algebras $\scr{B}$ over $Y$.
The second means that $\scr{B}$ is locally free over $Y$ (resp. means absolutely nothing) \emph{and}, further, that, for all $y\in Y$, the fibre $\scr{B}(y)=\scr{B}_y\otimes_{\cal{O}_y}k(y)$ is a separable algebra (i.e. a finite composition of finite separable extensions) over $k(y)$.

\begin{itenv}{Proposition 4.10}
\label{4.10}
  Let $X$ be a flat cover of $Y$ of degree $n$ (the definition of this term deserved to figure in \hyperref[I.4.9]{4.9}) defined by a locally free coherent sheaf $\scr{B}$ of algebras.
  We define, as usual, the trace homomorphism $\scr{B}\to\scr{A}$ (that is a homomorphism of $\scr{A}$-modules, where $\scr{A}=\cal{O}_Y$).
  For $X$ to be étale it is necessary and sufficient that the corresponding bilinear form $\tr_{\scr{B}/\scr{A}}xy$ define an isomorphism of $\scr{B}$ over $\scr{B}$, or, equivalently, that the \emph{discriminant section}
  \[
    d_{X/Y}
    = d_{\scr{B}/\scr{A}}
    \in \Gamma\big(Y,\wedge^n\check{\scr{B}}\otimes_\scr{A}\wedge^n\check{\scr{B}}\big)
  \]
  is invertible, or that the discriminant ideal defined by this section is the unit ring.
\end{itenv}

\begin{proof}
  We can reduce to the case where $Y=\Spec(k)$, and then it is a well-known criterion of separability, and thus trivial by passing to the algebraic closure of $k$.
\end{proof}

\begin{rmenv}{Remark}
  We will have a less trivial statement to make later on,
\oldpage{6}
  when we do not suppose a priori that $X$ is flat over $Y$, but instead require some normality hypothesis.
\end{rmenv}