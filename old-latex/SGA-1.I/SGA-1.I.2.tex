%!TEX root = ../SGA-1.tex
\section{Quasi-finite morphisms}
\label{I.2}

\begin{itenv}{Proposition 2.1}
\label{I.2.1}
  Let $A\to B$ be a local homomorphism (N.B. all rings are now Noetherian) and $\frak{m}$ the maximal ideal of $A$.
  Then the following conditions are equivalent:
  \begin{enumerate}[(i)]
    \item $B/\frak{m}B$ is of finite dimension over $k=A/\frak{m}$.
    \item $\frak{m}B$ is an ideal of definition, and $B/\frak{r}(B)=k(B)$ is an extension of $k=k(A)$.
    \item The completion $\hat{B}$ of $B$ is finite over the completion $\hat{A}$ of $A$.
  \end{enumerate}
\end{itenv}

\oldpage{2}
We then say that $B$ is \emph{quasi-finite} over $A$.
A morphism $f\colon X\to Y$ is said to be quasi-finite at $x$ (or the $Y$-prescheme $f$ is said to be quasi-finite at $x$) if $\cal{O}_x$ is quasi-finite over $\cal{O}_{f(x)}$.
This is equivalent to saying that $x$ is \emph{isolated in its fibre $f^{-1}(x)$}.
A morphism is said to be quasi-finite if it is quasi-finite at each point\footnote{In EGA~II~6.2.3 we further suppose that $f$ is of finite type.}.

\begin{itenv}{Corollary 2.2}
\label{I.2.2}
  If $A$ is complete, then quasi-finiteness is equivalent to finiteness.
\end{itenv}

We could also give the usual polysyllogism (i), (ii), (iii), (iv), (v) for quasi-finite morphisms, but that doesn't seem necessary here.
